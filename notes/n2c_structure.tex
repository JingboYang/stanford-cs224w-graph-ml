\section{Structure of a Graph}\label{s_2}
\subsection{Network Properties}\label{ss_21_network_properties}
\subsubsection{Degree Distribution}
For undirected graphs with $N$ nodes, let $N_k$ be the number of nodes with degree $k$. Plot normalized ($P(k) = \frac{N_k}{N}$) histogram of degree. It is common to plot with log-log scale.

\noindent For directed graphs, plot in-degree and out-degree distributions separately.

\subsubsection{Path Length}

\textbf{Path}: 
\begin{itemize}
    \item sequence of nodes in which each
node is linked to the next one
    \item can intersect itself
and pass through the
same edge multiple times
    \item follow the direction of arrows in directed graphs
\end{itemize}

\noindent \textbf{Distance}: number of edges along the shortest path between two nodes

\noindent \textbf{Diameter}: the maximum distance between any pair of nodes in a graph

\noindent \textbf{Average Path Length}: 
for a connected undirected graph (strongly connected directed graph):
\begin{itemize}
    \item Definition: $\bar{h} =  \frac{1}{2E_{max}}\sum_{i, j \neq i} h_{ij}$
    \item If the graph is not connected, we compute average only over the connected pairs (ignoring infinite length paths)
    \item We can also apply the measure to a (strongly) connected component of a graph
\end{itemize}
\subsubsection{Clustering Coefficient (for undirected graph)}

\noindent \textbf{Clustering Coefficient for a node $i$}:
\begin{itemize}
    \item measures how connected are $i$'s neighbors to each other.
    \item i has degree $k_i$ 
    \item $e_i$: number of edges are between the neighbors of node $i$
    \item Possible edges among $k_i$ neighbors of $i$'s: ${k_i \choose 2}$
    \item Define clustering coefficient of $i$: $C_i =  \frac{e_i}{{k_i \choose 2} }= \frac{2e_i}{k_i(k_i -1 )}$. 
    \item $C_i \in [0,1]$
\end{itemize}

\noindent \textbf{Average Clustering Coefficient}:
Take the average over all of the $N$ nodes of the graph
$C = \frac{1}{N} \sum_{i=1}^N C_i$
\subsubsection{Connected Component}

\noindent \textbf{Size of the largest connected component}: largest set where any two nodes can be connected via a path.

\noindent To compute the size of largest connected component, we need to find connected components following the below steps via BFS:
\begin{itemize}
    \item Start BFS from a random node
    \item Label the nodes BFS visited
    \item If all nodes are visited, the network is connected
    \item Otherwise, find an unvisited node and BFS from the node
\end{itemize}

\subsection{Random Graph Generation Models}
\subsubsection{Erdos-Renyi Model}
There are two variants of Erdos-Renyi Model: For an undirected graph of $n$ nodes
\begin{itemize}
    \item  \textit{variant 1} $G_{np}$: each edge $(u,v)$ appears \textit{i.i.d.} with probability $p$ (Similar to a Binomial distribution $Binomial(n-1,p)$) ($n-1$ since for every node, it is possible to have at most $n-1$ edges with other nodes).
    \item   \textit{variant 2} $G_{nm}$: $m$ edges picked uniformly at random 
\end{itemize}

\noindent The network properties (section \ref{ss_21_network_properties}) for a variant 1 $G_{np}$ Erdos-Renyi Model is as following:
\begin{itemize}
    \item Degree distribution: by properties of $Binomial(n-1,p)$
        \begin{itemize}
            \item $P(k) = {n-1 \choose k} p^k (1-p)^{n-1-k}$  
            \item By Law of Large Number, the distribution of average $P(k)$ becomes increasingly narrow as $n$ increases. We become increasingly confident that the degree of a node is in the vicinity of $k$ as $n$ grows.
        \end{itemize} 
    \item Clustering Coefficient:
        \begin{itemize}
            \item expected number of edges connecting to node $i$: $E(e_i) = \mathcal{P}(\text{a pair of neighbors of node } i \text{ is connected by an edge})$ $\times$ number of distinct pairs of neighbors of node $i$  $= p \times \frac{k_i (k_i -1)}{2}$
            \item expected clustering coefficient of node $i$ $E(C_i) = \frac{p \times k_i(k_i - 1)}{k_i(k_i - 1)} = p = \frac{\bar{k}}{n-1}$. 
        \end{itemize} 
    \item Path length: O(log(n))
        \begin{itemize}
                \item First we define \textbf{Expansion}: {\color{red} TODO }

                \item Expansion on random graphs  {\color{red} TODO }
                \item Average shortest path: If we keep $np$ constant, even when $G_{np}$ grows very large, the nodes will still be less than 20 hops (if we consider shortest path) apart on average.
            \end{itemize} 
    
    
    \item Connected Components:
    

\end{itemize}

\subsubsection{Small World Model}


\subsubsection{Kronecker Model}

\subsection{Subgraphs: Motifs and Graphlets}

\subsubsection{Definition}

\subsubsection{How to find motifs and graphlets}

\subsection{Structural Roles}

\subsubsection{How to find structural roles: RoIX}

\subsubsection{Application of structural roles}

\subsection{Spectral Clustering}\label{ss_21_spec_clus}