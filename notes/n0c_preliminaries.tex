\section{Preliminaries}

\subsection{Acknowledgement}\label{ss_01}

\subsection{Necessary Math}\label{ss_02}

\begin{todo}
We should cover
\begin{enumerate}
    \item Some matrix algebra (matrix multiplication, matrix derivative, eigenvalue, eigenvector, semi-definite)

    \item Probabilities (Bayes' rule, conditional independence, union bound)
    
    \item Basics on neural networks

\end{enumerate}{}

\end{todo}{}

\subsection{Other Relevant Courses}\label{ss_03}

Artificial intelligence in theory and in practice are connected to numerous sub-fields in computer science. As you might expect, contents taught in CS224W are also covered in other classes offered at Stanford. For your interest, and to our best knowledge, 

\paragraph{CS 265 Randomized Algorithms} goes in depth on probabilistic existence of edges, hence strongly related to spread of message (think disease transmission).

\paragraph{CS 261 A Second Course in Algorithms} goes in depth on traditional graphs (max-flow min-cut) along with some probabilistic components. With CS261 you'll develop a much better understanding of theoretical graph problems that solve real world problems.

\paragraph{CS 228 Probabilistic Graphical Networks} covers exactly what you think, Bayesian inference on graphs. This partially overlaps with CS265 and spends a considerable amount of time on message passing in graph. 

\paragraph{CS 229 Machine Learning} builds the foundation of machine learning. Though not directly relevant, it forms part of the traditional ML approach vs popular DL approach on data analysis.

\paragraph{CS 230 Deep Learning} is a great place to start if you are relatively new to deep learning. CS224W expects you to have decent knowledge in deep learning and all graph neural network techniques build on top of ``typical" deep learning approaches.

\paragraph{CS 246 Mining Massive Datasets} also deals with interconnected data. Page-rank is one of the main topics in CS246 for those interested in the inner-workings of a search engine.
